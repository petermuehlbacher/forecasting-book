\documentclass[11pt,reqno]{article}
\RequirePackage[OT1]{fontenc}
\RequirePackage{amsthm,amsmath}
\RequirePackage[numbers]{natbib}
\RequirePackage[colorlinks,linkcolor=blue,citecolor=blue,urlcolor=blue]{hyperref}
\usepackage{amssymb}
\usepackage{anysize}
\usepackage{hyperref}
\usepackage{extarrows}
\usepackage{multirow}
\usepackage{natbib}
\usepackage{stmaryrd}
\usepackage{color}
\usepackage{soul}
\usepackage{showlabels}
\usepackage{todonotes}
%\usepackage[disable]{todonotes}
\usepackage[perpage]{footmisc}



\numberwithin{equation}{section}
\newtheorem{thm}{Theorem}[section]
\newtheorem{lem}[thm]{Lemma}
\newtheorem{pro}[thm]{Proposition}
\newtheorem{cor}[thm]{Corollary}
\newtheorem{defi}[thm]{Definition}
\newtheorem{rem}[thm]{Remark}
\newtheorem{assu}[thm]{Assumption}
\newtheorem{Exa}{Example}[section]
\newcommand{\be}{\begin{equation}}
\newcommand{\ee}{\end{equation}}
\newcommand{\bea}{\begin{eqnarray*}}
\newcommand{\eea}{\end{eqnarray*}}
\renewcommand*{\bibfont}{\footnotesize}

\newcommand{\eqby}[1]{\mathrel{\stackrel{#1}{=}}}
\newcommand{\eqbydef}{\mathrel{\stackrel{\hspace*{-3mm}\text{\tiny(def)}\hspace*{-3mm}}{=}}}
\newcommand{\leqby}[1]{\mathrel{\stackrel{#1}{\leq}}}
\newcommand{\geqby}[1]{\mathrel{\stackrel{#1}{\geq}}}
\newcommand{\deq}{\mathrel{\mathop:}=}
\newcommand\numberthis{\addtocounter{equation}{1}\tag{\theequation}} % use as in http://tex.stackexchange.com/questions/42726/align-but-show-one-equation-number-at-the-end
%\begin{align*}
%a &=b \\
%  &=c \numberthis \label{eqn}
%\end{align*}

\newcommand{\iid}[1]{\mathrel{\stackrel{\text{iid}}{\sim}}#1}
\newcommand{\iidnormal}{\mathrel{\stackrel{\text{iid}}{\sim}}\mathcal N(0,1)}


\newcommand{\const}{\mbox{const}}

\newcommand{\eps}{\varepsilon}
\newcommand{\e}{\varepsilon}
\newcommand{\pt}{\partial}
\newcommand{\rd}{{\rm d}}


\newcommand{\bke}[1]{\left( #1 \right)}
\newcommand{\bkt}[1]{\left[ #1 \right]}
\newcommand{\bket}[1]{\left\{ #1 \right\}}
\newcommand{\norm}[1]{\| #1 \|}
\newcommand{\Norm}[1]{\left\Vert #1 \right\Vert}
\newcommand{\bka}[1]{\left\langle #1 \right\rangle}
\newcommand{\vect}[1]{\begin{bmatrix} #1 \end{bmatrix}}


\newcommand{\tr}{\mbox{tr\,}}


\renewcommand{\Re}{\mathsf{Re}\,}
\renewcommand{\Im}{\mathsf{Im}\,}
\newcommand{\supp}{\mathsf{supp}\,}

\newcommand{\E}{{\mathbb E }}
\newcommand{\R}{{\mathbb R }}
\newcommand{\N}{{\mathbb N}}
\newcommand{\Z}{{\mathbb Z}}
\renewcommand{\P}{{\mathbb P}}
\newcommand{\C}{{\mathbb C}}

\renewcommand{\div}{\mathop{\mathrm{div}}}
\newcommand{\curl}{\mathop{\mathrm{curl}}}
\newcommand{\wkto}{\rightharpoonup}


\newcommand{\Cr}{\color{red}}
\newcommand{\Cb}{\color{blue}}
\newcommand{\Cg}{\color{green}}
\newcommand{\nc}{\normalcolor}


\marginsize{35mm}{35mm}{38mm}{40mm}



\begin{document}
\title{Probabilistic Forecasting}
\date{\today}
%\author{Peter M\"{u}hlbacher}

\maketitle


%\begin{abstract}
%Abstract goes here. Lorem ipsum, blabla.
%\end{abstract}
\tableofcontents


\section{Introduction}


\section{Common Objections}

\subsection{``Saying `53\%' instead of `maybe' only creates a false sense of certainty"}
\subsection{``You can't put probabilities on one-off events"}
\subsection{``This trajectory of forecasts looks wrong to me, these forecasts can't be any good"}

\section{Scoring Methods}

\section{Aggregation of Forecasts}
Note on prediction markets, subject matter experts vs good forecasters, \dots

\subsection{Theoretical Results}

\subsection{Empirical Observations}


\section{Model Building}
\subsection{Useful Priors}

\subsection{Model Uncertainty}

\subsection{Maximum Entropy Principle}

\subsection{Distributions From Universality/Fixed-Point Considerations}

\appendix
\section{Relevant Bits of Probability Theory}
\subsection{Basics}
Probability measures, (conditional) expectations (both on events and on $\sigma$-algebra, (conditional) probability, \dots
\subsection{Martingales}
\subsection{Maximum Entropy Principle}
\section{Causality}
Why is causality important? $\rightarrow$ model extrapolation




\bibliography{references}{}
\bibliographystyle{alpha}


\end{document}